\documentclass[12pt,a4paper]{report}
\usepackage[utf8]{vietnam}
\usepackage{geometry}
\usepackage{fancyhdr}
\usepackage{hyperref}
\usepackage{listings}
\usepackage{graphicx}
\usepackage{subcaption}
\usepackage{float}
\usepackage{xcolor}
\usepackage{amsmath}
\usepackage{titlesec}
\usepackage{tocloft}
\usepackage{mathptmx}
\usepackage{longtable}

\usepackage{tikz}
\usetikzlibrary{calc}
% ------------------------------------

\graphicspath{ {./img/} }

\lstdefinelanguage{json}{
  morestring=[b]",
  morecomment=[l]{//},
  morekeywords={true,false,null},
  sensitive=false,
  alsoletter={:},
  moredelim=[l][\color{black}\bfseries]{"},
  moredelim=[s][\color{black}\bfseries]{:}{,},
}

\geometry{
  a4paper,
  left=3cm,
  right=2cm,
  top=2cm,
  bottom=2cm
}

\pagestyle{fancy}
\fancyhf{}
\fancyhead[L]{\leftmark}
\fancyfoot[C]{\thepage}

\hypersetup{
  colorlinks=true,
  linkcolor=black,
  filecolor=blue,
  urlcolor=blue,
  citecolor=black
}

% Định dạng code
\lstset{
  basicstyle=\ttfamily\small,
  breaklines=true,
  frame=single,
  language=Java,
  showstringspaces=false,
  commentstyle=\color{gray},
  keywordstyle=\color{blue},
  stringstyle=\color{red}
}

\begin{document}

%----------- TRANG BÌA -----------
\begin{titlepage}
    \begin{tikzpicture}[overlay,remember picture]
        \draw [line width=3pt]
            ($ (current page.north west) + (3.0cm,-2.0cm) $)
            rectangle
            ($ (current page.south east) + (-2.0cm,2.0cm) $);
        \draw [line width=1pt]
            ($ (current page.north west) + (3.15cm,-2.15cm) $)
            rectangle
            ($ (current page.south east) + (-2.15cm,2.15cm) $);
    \end{tikzpicture}

    \begin{center}
        % \textbf{\large BỘ GIAO THÔNG VẬN TẢI}\\
        \textbf{\large TRƯỜNG ĐẠI HỌC GIAO THÔNG VẬN TẢI}\\[0.2cm] 
        \textbf{\large TP. HỒ CHÍ MINH}\\[0.5cm]
        \textbf{\large Viện Công nghệ thông tin và Điện, Điện tử}\\[2cm]
        
        \includegraphics[width=10cm, keepaspectratio]{./img/ut-logo.png}\\[2cm]
        
        \textbf{\LARGE BÁO CÁO ĐỒ ÁN MÔN HỌC}\\[0.5cm]
        \textbf{\Large LẬP TRÌNH THIẾT BỊ DI ĐỘNG}\\[1.5cm]
        
        \textbf{\huge SALUD}\\[0.5cm]
        \textbf{\Large Ứng dụng Quản lý Sức khỏe Cá nhân}\\[2.5cm]
        
        \begin{table}[h]
            \centering
            \begin{tabular}{r l}
                \textbf{Giảng viên hướng dẫn:} & Thầy Trương Quang Tuấn \\[0.5cm]
                \textbf{Sinh viên thực hiện:} & Nguyễn Đức Duy - 083205012971 \\
                               & Võ Văn Hiền \hspace{0.70cm} - 087205002322 \\
            \end{tabular}
        \end{table}
        
        \vfill
        
        % Ngày tháng
        \textbf{Tp. Hồ Chí Minh, Tháng 12 năm 2025}
    \end{center}
\end{titlepage}



%----------- MỤC LỤC -----------
\tableofcontents
\newpage

%----------- CHƯƠNG 1: THÔNG TIN NHÓM -----------
\chapter{THÔNG TIN NHÓM}

\section{Thành viên nhóm}
\begin{table}[h]
\centering
\begin{tabular}{|c|l|c|p{6cm}|}
\hline
\textbf{STT} & \textbf{Họ và Tên} & \textbf{MSSV} & \textbf{Vai trò} \\ \hline

1 & Nguyễn Đức Duy & 083205012971 &
Team Member, UI/UX Designer, \newline Frontend Developer, \newline Backend Developer. \\ \hline

2 & Võ Văn Hiền & 087205002322 &
Team Member, UI/UX Designer, \newline Frontend Developer, \newline Backend Developer .\\ \hline

\end{tabular}
\caption{Danh sách thành viên nhóm}
\end{table}


\section{Phân công công việc}

\subsection{Nguyễn Đức Duy}
\begin{itemize}
\item Thiết kế giao diện người dùng bằng Figma
\item Thiết kế kiến trúc ứng dụng (MVVM)
\item Xây dựng các màn hình và điều hướng (Navigation, Bottom Tab)
\item Tích hợp Firebase Authentication và Firestore
\item Phát triển tính năng quản lý sức khỏe (theo dõi cân nặng, chiều cao, BMI) và quản lý dinh dưỡng
\item Phát triển tính năng luyện tập và nhật ký sức khỏe
\item Tạo các component tái sử dụng
\item Testing và tối ưu hóa hiệu năng, giao diện người dùng
\item Viết báo cáo, tài liệu kỹ thuật 
\end{itemize}

\subsection{Võ Văn Hiền}
\begin{itemize}
\item Thiết kế giao diện người dùng bằng Figma
\item Thiết kế kiến trúc ứng dụng (MVVM)
\item Xây dựng các màn hình và điều hướng (Navigation, Bottom Tab)
\item Tích hợp Firebase Authentication và Firestore
\item Phát triển tính năng quản lý sức khỏe (theo dõi cân nặng, chiều cao, BMI) và quản lý dinh dưỡng
\item Phát triển tính năng luyện tập và nhật ký sức khỏe
\item Tạo các component tái sử dụng
\item Testing và tối ưu hóa hiệu năng, giao diện người dùng
\item Viết slide trình bày
\end{itemize}

%----------- CHƯƠNG 2: BỐI CẢNH VÀ LÝ DO -----------
\chapter{BỐI CẢNH VÀ LÝ DO CHỌN ĐỀ TÀI}

\section{Bối cảnh}
Trong xã hội hiện đại, việc chăm sóc sức khỏe ngày càng được quan tâm. Tuy nhiên, nhiều người gặp khó khăn trong việc theo dõi và duy trì lối sống lành mạnh do:
\begin{itemize}
    \item Cuộc sống bận rộn, thiếu thời gian tự theo dõi sức khỏe
    \item Thiếu kiến thức về dinh dưỡng và luyện tập phù hợp
    \item Không có công cụ hỗ trợ hiệu quả và trực quan
    \item Khó khăn trong việc duy trì động lực và theo dõi tiến trình
\end{itemize}

\section{Lý do chọn đề tài}
Nhóm chọn phát triển ứng dụng \textbf{Salud} (tiếng Tây Ban Nha có nghĩa là "sức khỏe") vì những lý do sau:

\begin{enumerate}
    \item \textbf{Nhu cầu thực tế:} Người dùng cần một công cụ "tất cả trong một" để quản lý sức khỏe tổng quan.
    \item \textbf{Xu hướng công nghệ:} Ứng dụng di động đang là xu hướng chăm sóc sức khỏe cá nhân.
    \item \textbf{Tính ứng dụng cao:} Phù hợp với đa dạng đối tượng người dùng từ sinh viên đến người đi làm.
    \item \textbf{Học hỏi công nghệ:} Cơ hội áp dụng các công nghệ hiện đại như Kotlin, Jetpack Compose, Firebase.
\end{enumerate}

\section{Mục tiêu của ứng dụng}
\begin{itemize}
    \item Cung cấp nền tảng quản lý sức khỏe toàn diện và dễ sử dụng
    \item Giúp người dùng theo dõi các chỉ số cơ thể theo thời gian
    \item Hỗ trợ quản lý dinh dưỡng và tính toán calo khoa học
    \item Tạo động lực và giúp người dùng đạt mục tiêu sức khỏe
    \item Cung cấp giao diện thân thiện và trải nghiệm người dùng tốt
\end{itemize}

%----------- CHƯƠNG 3: PHẠM VI NGHIÊN CỨU -----------
\chapter{PHẠM VI NGHIÊN CỨU}

\section{Phạm vi của dự án}

\subsection{Giới hạn nội dung}
Dự án tập trung vào việc xây dựng ứng dụng di động Android với các tính năng cốt lõi:
\begin{itemize}
    \item Quản lý tài khoản người dùng (đăng nhập, đăng ký)
    \item Dashboard hiển thị tổng quan sức khỏe
    \item Theo dõi các chỉ số cơ thể (cân nặng, chiều cao, BMI)
    \item Quản lý dinh dưỡng và calo
    \item Nhật ký sức khỏe hằng ngày
    \item Tích hợp AI hỗ trợ báo cáo tình trạng sức khỏe và gọi ý phù hợp theo yêu cầu
    \item Quản lý hồ sơ cá nhân và mục tiêu
\end{itemize}

\subsection{Giới hạn kỹ thuật}
\begin{itemize}
    \item \textbf{Nền tảng:} Android (API Level 25+)
    \item \textbf{Ngôn ngữ:} Kotlin
    \item \textbf{Framework UI:} Jetpack Compose
    \item \textbf{Backend:} Firebase (Authentication, Firestore)
    \item \textbf{Kiến trúc:} MVVM (Model-View-ViewModel)
\end{itemize}

\subsection{Giới hạn thời gian}
Dự án được thực hiện trong thời gian 3 tháng (từ tháng 9 đến tháng 12 năm 2025).

\section{Đối tượng người dùng}
\begin{enumerate}
    \item \textbf{Sinh viên \& Nhân viên văn phòng:} Người bận rộn cần công cụ nhanh gọn để theo dõi sức khỏe.
    \item \textbf{Người tập gym, fitness:} Người muốn theo dõi chế độ ăn và lượng calo chi tiết.
    \item \textbf{Người quan tâm sức khỏe:} Người muốn theo dõi các chỉ số cơ thể định kỳ để phòng ngừa bệnh tật.
\end{enumerate}

%----------- CHƯƠNG 4: PHƯƠNG PHÁP NGHIÊN CỨU -----------
\chapter{PHƯƠNG PHÁP NGHIÊN CỨU}

\section{Phương pháp thu thập dữ liệu}

\subsection{Nghiên cứu lý thuyết}
\begin{itemize}
    \item Tìm hiểu về các ứng dụng quản lý sức khỏe hiện có (Google Fit, Samsung Health)
    \item Nghiên cứu các công nghệ: Kotlin, Jetpack Compose, Firebase, MVVM
\end{itemize}

\subsection{Phân tích}
\begin{itemize}
    \item Phân tích điểm mạnh/yếu của các ứng dụng tương tự
    \item Xác định các tính năng quan trọng nhất theo người dùng
\end{itemize}

\section{Quy trình phát triển}
Nhóm áp dụng mô hình \textbf{Agile - Scrum} với các bước:

\subsection{Sprint 1 (Tuần 1-3): Lập kế hoạch và Thiết kế}
\begin{itemize}
    \item Phân tích yêu cầu chi tiết
    \item Thiết kế Database Schema
    \item Thiết kế giao diện trên Figma
    \item Xây dựng kiến trúc ứng dụng
\end{itemize}

\subsection{Sprint 2 (Tuần 4-6): Phát triển Core Features}
\begin{itemize}
    \item Xây dựng Authentication (Login/Register)
    \item Phát triển Dashboard và Navigation
    \item Tích hợp Firebase
\end{itemize}

\subsection{Sprint 3 (Tuần 7-9): Phát triển tính năng chính}
\begin{itemize}
    \item Quản lý sức khỏe (Weight, Height, BMI)
    \item Quản lý dinh dưỡng
    \item Nhật ký sức khỏe
\end{itemize}

\subsection{Sprint 4 (Tuần 10-12): Testing và Hoàn thiện}
\begin{itemize}
    \item Unit Testing và Integration Testing
    \item UI/UX Testing
    \item Bug fixing và tối ưu hóa
    \item Viết tài liệu và báo cáo
\end{itemize}

\section{Công cụ và công nghệ}

\subsection{Công cụ phát triển}
\begin{itemize}
    \item \textbf{IDE:} Android Studio (Narwhal 3 Feature Drop | 2025.1.3)
    \item \textbf{Version Control:} GitHub
    \item \textbf{Design Tool:} Figma
    \item \textbf{Project Management:} Báo cáo tiến độ qua Zalo, trao đổi trực tuyến qua Discord.
\end{itemize}

\subsection{Công nghệ sử dụng}

\subsection{Công nghệ sử dụng}

\begin{longtable}{|p{4cm}|p{5.5cm}|p{5.5cm}|}
\caption{Các công nghệ cốt lõi và nổi bật trong ứng dụng Salud} \label{tab:technologies} \\
\hline
\textbf{Công nghệ} & \textbf{Chức năng} & \textbf{Nơi sử dụng} \\ \hline
\endfirsthead

\hline
\textbf{Công nghệ} & \textbf{Chức năng} & \textbf{Nơi sử dụng} \\ \hline
\endhead

\hline
\endfoot

\hline
\endlastfoot

\multicolumn{3}{|c|}{\textbf{Ngôn ngữ \& Framework}} \\ \hline
Kotlin & Ngôn ngữ lập trình chính với Coroutines, Null-safety và Extension Functions & Toàn bộ source code ứng dụng \\ \hline
Jetpack Compose & Modern UI Toolkit declarative, thay thế XML layouts truyền thống & Tất cả màn hình: Home, Profile, Data, Diary, Sign-in \\ \hline
Android SDK & Nền tảng phát triển ứng dụng Android native & Hỗ trợ Android 7.1+ (API 25-36) \\ \hline

\multicolumn{3}{|c|}{\textbf{Google AI \& Machine Learning}} \\ \hline
Gemini AI (Google) & Large Language Model - AI chatbot tư vấn sức khỏe thông minh & HintScreen (tư vấn dinh dưỡng, tập luyện), NutritionScreen (phân tích thành phần món ăn) \\ \hline

\multicolumn{3}{|c|}{\textbf{Firebase Platform}} \\ \hline
Firebase Authentication & Xác thực đa phương thức: Email/Password, Google Sign-In & AuthViewModel, SignInScreen, SplashScreen \\ \hline
Cloud Firestore & NoSQL Database realtime với offline sync & Lưu trữ User, HealthRecord, Exercise, Nutrition, Sleep, SavedChat \\ \hline
Firebase Cloud Messaging & Push Notifications đa nền tảng & Nhắc nhở uống nước, tập thể dục, ghi chép sức khỏe \\ \hline

\multicolumn{3}{|c|}{\textbf{Google APIs \& Services}} \\ \hline
Google Sign-In API & One-tap đăng nhập bằng tài khoản Google & SignInScreen với Credential Manager \\ \hline
Google Identity Services & Quản lý xác thực và credentials an toàn & Lưu trữ token, auto sign-in \\ \hline
Google Play Services & Tích hợp các dịch vụ Google & Authentication, Location, Activity Recognition \\ \hline

\multicolumn{3}{|c|}{\textbf{Android Sensors}} \\ \hline
Step Counter Sensor & Đếm số bước chân realtime & HomeScreen - theo dõi hoạt động hàng ngày \\ \hline
Activity Recognition API & Nhận diện hoạt động: đi bộ, chạy, đạp xe & Tự động ghi nhận bài tập \\ \hline
Location Services & Xác định vị trí, tính quãng đường di chuyển & Tracking quãng đường tập luyện \\ \hline

\multicolumn{3}{|c|}{\textbf{Architecture Components}} \\ \hline
Navigation Compose & Type-safe navigation giữa các màn hình & NavGraph với NavHost, deep linking \\ \hline
ViewModel + StateFlow & MVVM pattern, reactive state management & HomeViewModel, AuthViewModel, HintViewModel, NutritionViewModel \\ \hline
Kotlin Coroutines & Asynchronous programming, structured concurrency & Gọi Firebase, Gemini AI, xử lý background tasks \\ \hline

\multicolumn{3}{|c|}{\textbf{UI/UX Components}} \\ \hline
Material Design 3 & Hệ thống thiết kế hiện đại & Theme, Typography, ColorScheme, Components \\ \hline
Material Icons Extended & Bộ icon phong phú 2000+ icons & Icons xuyên suốt ứng dụng \\ \hline
Coil & Image loading \& caching hiệu suất cao & AsyncImage - Avatar, hình ảnh món ăn \\ \hline
Canvas API & Vẽ đồ họa tùy chỉnh & Biểu đồ vòng tròn tiến độ, Line Chart \\ \hline

\multicolumn{3}{|c|}{\textbf{Data \& Storage}} \\ \hline
Kotlin Serialization & JSON parsing type-safe & Parse response Gemini AI, data models \\ \hline

\multicolumn{3}{|c|}{\textbf{Security \& Permissions}} \\ \hline
Accompanist Permissions & Runtime permissions handling & Location, Notification, Activity Recognition permissions \\ \hline

\end{longtable}



%----------- CHƯƠNG 5: CÁC TÍNH NĂNG CHÍNH -----------
\chapter{CÁC TÍNH NĂNG CHÍNH}

\section{Sơ đồ tổng quan}
Ứng dụng Salud bao gồm các module chính:
\begin{itemize}
    \item Authentication Module
    \item Health Tracking Module
    \item Nutrition Management Module
    \item Diary Module
    \item Profile \& Settings Module
\end{itemize}

\section{Tính năng chi tiết}

\subsection{Quản lý Tài khoản}
\textbf{Mô tả:} Cho phép người dùng đăng ký, đăng nhập an toàn để cá nhân hóa trải nghiệm.

\textbf{Chức năng:}
\begin{itemize}
    \item Đăng ký tài khoản mới với email và mật khẩu
    \item Đăng nhập bằng email/password
    \item Xác thực người dùng qua Firebase Authentication
    \item Lưu trữ thông tin người dùng trên Firestore
    \item Quản lý phiên đăng nhập
\end{itemize}

\textbf{Trạng thái:} Hoàn thành

\subsection{Dashboard Tổng quan}
\textbf{Mô tả:} Hiển thị các chỉ số quan trọng ngay màn hình chính: cân nặng, BMI, calo trong ngày.

\textbf{Chức năng:}
\begin{itemize}
    \item Hiển thị chào mừng người dùng
    \item Thống kê nhanh các chỉ số: Weight, Height, BMI
    \item Hiển thị calo tiêu thụ trong ngày
    \item Card nhanh để truy cập các chức năng
    \item Biểu đồ xu hướng sức khỏe
\end{itemize}

\textbf{Trạng thái:} Hoàn thành

\subsection{Theo dõi Sức khỏe}
\textbf{Mô tả:} Ghi nhận và trực quan hóa các chỉ số (cân nặng, chiều cao, BMI) qua biểu đồ theo thời gian.

\textbf{Chức năng:}
\begin{itemize}
    \item Ghi nhận cân nặng (Weight) hằng ngày
    \item Theo dõi chiều cao (Height)
    \item Tự động tính toán BMI
    \item Hiển thị biểu đồ xu hướng theo thời gian
    \item Phân loại tình trạng sức khỏe (Thiếu cân, Bình thường, Thừa cân)
    \item Lưu trữ lịch sử dữ liệu
\end{itemize}

\textbf{Công thức BMI:}
\begin{equation}
BMI = \frac{Weight (kg)}{Height^2 (m^2)}
\end{equation}

\textbf{Trạng thái:} Hoàn thành

\subsection{Quản lý Dinh dưỡng}
\textbf{Mô tả:} Theo dõi lượng calo nạp vào từ các bữa ăn, tìm kiếm thực phẩm và xây dựng thực đơn.

\textbf{Chức năng:}
\begin{itemize}
    \item Ghi nhận bữa ăn (Sáng, Trưa, Tối, Bữa chính, Bữa phụ)
    \item Tính toán tổng calo trong ngày
    \item So sánh calo tiêu thụ với mục tiêu
    \item Phân tích macro nutrients (Protein, Carb, Fat) qua Google Gemini
\end{itemize}

\textbf{Trạng thái:} Hoàn thành

\subsection{Quản lý Vận động}
\textbf{Mô tả:} Ghi nhận các hoạt động thể chất, theo dõi thời lượng và lượng calo tiêu thụ.

\textbf{Chức năng:}
\begin{itemize}
    \item Ghi nhận các bài tập (Chạy, Đạp xe, Gym, Yoga...)
    \item Tính toán calo đốt cháy
    \item Theo dõi thời lượng tập luyện
    \item Lịch sử hoạt động
    \item Thống kê tuần/tháng
\end{itemize}

\textbf{Trạng thái:} Hoàn thành

\subsection{Theo dõi Giấc ngủ}
\textbf{Mô tả:} Ghi nhận thời gian ngủ và thức dậy, đánh giá chất lượng giấc ngủ.

\textbf{Chức năng:}
\begin{itemize}
    \item Ghi nhận giờ đi ngủ và thức dậy
    \item Tính toán tổng thời gian ngủ
    \item Đánh giá chất lượng giấc ngủ
    \item Biểu đồ theo dõi giấc ngủ
\end{itemize}

\textbf{Trạng thái:} Hoàn thành

\subsection{Thiết lập Mục tiêu}
\textbf{Mô tả:} Đặt ra các mục tiêu sức khỏe và theo dõi tiến trình thực hiện.

\textbf{Chức năng:}
\begin{itemize}
    \item Đặt mục tiêu giảm/tăng cân
    \item Đặt mục tiêu calo hằng ngày
    \item Theo dõi tiến độ đạt mục tiêu
\end{itemize}

\textbf{Trạng thái:} Hoàn thành

\subsection{Nhật ký Sức khỏe}
\textbf{Mô tả:} Ghi chép các hoạt động sức khỏe hằng ngày dưới dạng diary.

\textbf{Chức năng:}
\begin{itemize}
    \item Xem lịch theo ngày
    \item Xem tổng hợp hoạt động trong ngày
\end{itemize}

\textbf{Trạng thái:} Hoàn thành

\subsection{Quản lý Hồ sơ}
\textbf{Mô tả:} Quản lý thông tin cá nhân và cài đặt ứng dụng.

\textbf{Chức năng:}
\begin{itemize}
    \item Cập nhật thông tin cá nhân
    \item Đổi mật khẩu
    \item Đăng xuất
\end{itemize}

\textbf{Trạng thái:} Hoàn thành

\subsection{Trợ lý AI Sức khỏe}
\textbf{Mô tả:} Hệ thống AI thông minh cung cấp tư vấn, hỏi đáp về sức khỏe và đưa ra gợi ý cá nhân hóa.

\textbf{Chức năng:}
\begin{itemize}
    \item Chatbot AI trả lời câu hỏi về sức khỏe, dinh dưỡng, luyện tập
    \item Phân tích dữ liệu sức khỏe cá nhân và đưa ra đánh giá
    \item Phân tích dinh dưỡng tự động
    \item Gợi ý chế độ ăn uống phù hợp dựa trên mục tiêu
    \item Gợi ý bài tập thể dục phù hợp với thể trạng
    \item Cảnh báo khi phát hiện bất thường về sức khỏe
    \item Tips và tricks hằng ngày về lối sống lành mạnh
    \item Trả lời bằng ngôn ngữ tự nhiên, dễ hiểu
\end{itemize}

\textbf{Công nghệ AI:}
\begin{itemize}
    \item Sử dụng Google Gemini AI để xử lý ngôn ngữ tự nhiên
    \item Model AI: Gemini 2.0 Flash Lite - Phiên bản nhẹ, tốc độ cao của Google Gemini
\end{itemize}

\textbf{Ví dụ câu hỏi:}
\begin{itemize}
    \item "Gợi ý bữa ăn"
    \item "Cho tôi kế hoạch tập luyện"
    \item "BMI của tôi có bình thường không?"
    \item "Làm sao để cải thiện chất lượng giấc ngủ?"
    \item "Thực đơn cho người tiểu đường như thế nào?"
\end{itemize}

\textbf{Trạng thái:} Đang hoàn thiện

%----------- CHƯƠNG 6: THIẾT KẾ HỆ THỐNG -----------
\chapter{THIẾT KẾ HỆ THỐNG}

\section{Kiến trúc ứng dụng}
Ứng dụng Salud được xây dựng theo mô hình \textbf{MVVM (Model-View-ViewModel)}.

\subsection{Mô hình MVVM}
\begin{itemize}
    \item \textbf{Model:} Chứa business logic và data layer (Room Database, Firebase)
    \item \textbf{View:} Giao diện người dùng (Jetpack Compose Screens)
    \item \textbf{ViewModel:} Quản lý UI state và xử lý logic giữa View và Model
\end{itemize}

\subsection{Cấu trúc thư mục}
\begin{lstlisting}
app/src/main/java/com/example/salud_app/
|-- model/              # Data classes, Entities
|-- ui/
|   |-- screen/         # Cac man hinh
|   |   |-- sign/       # SignIn, SignUp
|   |   |-- home/       # HomeScreen, HomeViewModel
|   |   |-- data/       # Data tracking screens
|   |   |-- diary/      # DiaryScreen, DiaryViewModel
|   |   |-- profile/    # ProfileScreen, ProfileViewModel
|   |-- components/     # Reusable UI components
|   |-- theme/          # Theme, Colors, Typography
|-- navigation/         # Navigation setup
|-- MainActivity.kt     # Entry point
\end{lstlisting}

\section{Thiết kế Database}

\subsection{Firebase Firestore Collections}
\begin{enumerate}
    \item \textbf{users}: Thông tin người dùng
    \begin{lstlisting}[language=json]
{
  "userId": "string",
  "fullName" : "string",
  "birthDate" : "string",
  "gender" : "string",
  "numPhone" : "string",
  "email" : "string",
  "photoUrl" : "string",
  "updateAt" : "number",
  "createAt" : number
  

  
}
    \end{lstlisting}
    
    \item \textbf{health\_records}: Dữ liệu sức khỏe
    \begin{lstlisting}[language=json]
{
  "bmi": "number",
  "date": "string",
  "diastolic": "number",
  "weight": "number",
  "height": "number",
  "bmi": "number",
  "bloodPressure": "string",
  "heartRate": "number",
  "timestamp": "number",
  "userId: "string
}
    \end{lstlisting}
    
    \item \textbf{nutrition\_logs}: Nhật ký dinh dưỡng
    \begin{lstlisting}[language=json]
{
  "logId": "string",
  "userId": "string",
  "date": "timestamp",
  "mealType": "string",
  "mealCategory": "string",
  "mealName": "string",
  "calories": "number",
  "protein": "number",
  "carbs": "number",
  "fat": "number",
  "time": "string",
}
    \end{lstlisting}
    
    \item \textbf{goals}: Mục tiêu cá nhân
    \begin{lstlisting}[language=json]
{
  "goalId": "string",
  "userId": "string",
  "targetWeight": "number",
  "targetCalories": "number",
  "startDate": "timestamp",
  "endDate": "timestamp",
  "status": "string"
}
    \end{lstlisting}
\end{enumerate}

\section{Thiết kế giao diện}

\subsection{Design System}
\begin{itemize}
    \item \textbf{Color Scheme:} 
    \begin{itemize}
        \item Primary: DeepSkyBlue (\#0xFF2868D0)
        \item Secondary: SkyBlueLight (\#0xFFA8CEFA)
        \item Background: White/Light Gray
        \item Text: Dark Gray/Black
    \end{itemize}
    \item \textbf{Typography:} 
    \begin{itemize}
        \item Font Family: Roboto, San Francisco
        \item Heading: Bold, 24-32sp
        \item Body: Regular, 14-16sp
    \end{itemize}
    \item \textbf{Components:} Cards, Buttons, Input Fields, Charts
\end{itemize}

\subsection{Màn hình chính}
\begin{enumerate}
    \item \textbf{Splash Screen}: Màn hình chào mừng với logo và animation
    \item \textbf{Sign In/Sign Up}: Đăng nhập và đăng ký
    \item \textbf{Home Screen}: Dashboard tổng quan
    \item \textbf{Data Screen}: Theo dõi dữ liệu sức khỏe
    \item \textbf{Diary Screen}: Nhật ký sức khỏe
    \item \textbf{Profile Screen}: Quản lý hồ sơ cá nhân
\end{enumerate}

Link Figma: \url{https://www.figma.com/design/siycXmpdM1pq3lDUAJnDZz/Mobile-App-UI}

\section{Luồng dữ liệu}
\begin{enumerate}
    \item User tương tác với View (Compose Screen)
    \item View gọi hàm trong ViewModel
    \item ViewModel xử lý logic và gọi Repository/Firebase
    \item Dữ liệu được lấy từ Firebase
    \item ViewModel cập nhật State
    \item View tự động re-compose khi State thay đổi
\end{enumerate}

%----------- CHƯƠNG 7: TRIỂN KHAI -----------
\chapter{TRIỂN KHAI VÀ KẾT QUẢ}

\section{Môi trường phát triển}
\begin{itemize}
    \item \textbf{OS:} Windows 11
    \item \textbf{IDE:} Android Studio Narwhal 3 Feature Drop | 2025.1.3
    \item \textbf{JDK:} Java 11
    \item \textbf{Gradle:} 8.13
    \item \textbf{Min SDK:} 25 (Android 7.1)
    \item \textbf{Target SDK:} 36 (Android 16)
\end{itemize}

\section{Các ViewModel đã triển khai}

\subsection{HomeViewModel}
Quản lý dữ liệu và logic cho màn hình Home:
\begin{itemize}
    \item Load thông tin người dùng
    \item Hiển thị các chỉ số sức khỏe tổng quan
    \item Tính toán BMI
    \item Hiển thị calo trong ngày
\end{itemize}

\subsection{AuthViewModel (SignInViewModel)}
Quản lý xác thực người dùng:
\begin{itemize}
    \item Xử lý đăng nhập với Firebase
    \item Xử lý đăng ký tài khoản mới
    \item Quản lý trạng thái đăng nhập
    \item Xử lý lỗi authentication
\end{itemize}

\subsection{WeightViewModel, HeightViewModel, BMIViewModel}
Quản lý dữ liệu sức khỏe:
\begin{itemize}
    \item CRUD operations cho dữ liệu cân nặng, chiều cao
    \item Tính toán BMI tự động
    \item Lưu trữ lịch sử dữ liệu
    \item Hiển thị biểu đồ xu hướng
\end{itemize}

\subsection{NutritionViewModel}
Quản lý dinh dưỡng:
\begin{itemize}
    \item Thêm/sửa/xóa bữa ăn
    \item Tính toán tổng calo
    \item Quản lý danh sách thực phẩm
    \item Theo dõi macro nutrients
\end{itemize}

\subsection{DiaryViewModel}
Quản lý nhật ký:
\begin{itemize}
    \item Hiển thị lịch
    \item Load dữ liệu theo ngày
    \item Thống kê hoạt động
\end{itemize}

\subsection{ProfileViewModel}
Quản lý hồ sơ:
\begin{itemize}
    \item Cập nhật thông tin cá nhân
    \item Xử lý đăng xuất
\end{itemize}

\subsection{SleepViewModel}
Quản lý giấc ngủ
\begin{itemize}
    \item Thêm/sửa/xóa giấc ngủ
    \item Tính toán thời gian ngủ
    \item Quản lý danh sách giấc ngủ
\end{itemize}

\subsection{HintViewModel}
Quản lý trợ lý AI:
\begin{itemize}
    \item Xử lý câu hỏi từ người dùng
    \item Gọi API AI để nhận câu trả lời
    \item Lưu trữ lịch sử hội thoại
    \item Phân tích dữ liệu sức khỏe để đưa ra gợi ý
    \item Tạo insights và recommendations cá nhân hóa
\end{itemize}

\section{Các Screen đã triển khai}



\subsection{Core Screens}
\begin{itemize}
    \item \textbf{SplashScreen}: Màn hình khởi động với animation
    \item \textbf{LoginScreen (SignInScreen)}: Giao diện đăng nhập
    \item \textbf{HomeScreen}: Dashboard chính
    \item \textbf{DataScreen}: Màn hình quản lý dữ liệu
    \item \textbf{DiaryScreen}: Nhật ký sức khỏe
    \item \textbf{ProfileScreen}: Hồ sơ cá nhân
    \item \textbf{HintScreen}: Trợ lý AI
\end{itemize}

\subsection{Data Tracking Screens}
\begin{itemize}
    \item \textbf{DataHealthScreen}: Tổng quan sức khỏe
    \item \textbf{DataHealthWeightScreen}: Theo dõi cân nặng
    \item \textbf{DataHealthHeightScreen}: Theo dõi chiều cao
    \item \textbf{DataHealthBMIScreen}: Theo dõi BMI
    \item \textbf{DataNutritionScreen}: Quản lý dinh dưỡng
    \item \textbf{DataHintScreen}: Gợi ý và tips
\end{itemize}

\section{Navigation System}
Ứng dụng sử dụng \textbf{Navigation Compose} với Bottom Navigation Bar:
\begin{itemize}
    \item Tab Home: Màn hình chính
    \item Tab Data: Quản lý dữ liệu sức khỏe
    \item Tab Diary: Nhật ký
    \item Tab Profile: Hồ sơ cá nhân
    \item Tab Hint: Trợ lý AI
    
\end{itemize}

\section{Firebase Integration}

\subsection{Firebase Authentication}
\begin{itemize}
    \item Đăng ký với Email/Password
    \item Đăng nhập với Email/Password
    \item Đăng nhập với Google
    \item Quản lý session
    \item Password reset (qua email)
\end{itemize}

\subsection{Firebase Firestore}
\begin{itemize}
    \item Lưu trữ dữ liệu người dùng
    \item Lưu trữ health records
    \item Lưu trữ nutrition logs
    \item Real-time synchronization
\end{itemize}

\section{Kết quả đạt được}

\subsection{Các tính năng hoàn thành}
\begin{itemize}
    \item Authentication system hoàn chỉnh
    \item Dashboard tổng quan trực quan
    \item Theo dõi cân nặng, chiều cao, BMI
    \item Giao diện người dùng đẹp mắt với Jetpack Compose
    \item Navigation system mượt mà
    \item Quản lý state hiệu quả với ViewModel
    \item Tích hợp Firebase thành công
\end{itemize}

\subsection{Đánh giá tiến độ so với thiết kế}
Ứng dụng được hoàn thành khoảng 90\% so với thiết kế ban đầu trên Figma, trong đó có phát triển thêm trong quá trình xây dựng và phát triển dự án.

\begin{figure}[H] 
        \centering
        \includegraphics[width=15cm]{img/Mobile App UI.png}
        \caption{Thiết kế ban đầu trên Figma}
        \label{fig:qr_code}
    \end{figure}


\subsection{Hình ảnh ứng dụng}

\begin{figure}[H]
    \centering
    
    \begin{subfigure}[b]{0.23\textwidth}
        \centering
        \includegraphics[width=\textwidth]{splash_screen.jpg}
        \caption{Màn hình chào}
    \end{subfigure}
    \hfill
    \begin{subfigure}[b]{0.23\textwidth}
        \centering
        \includegraphics[width=\textwidth]{dash1.jpg}
        \caption{Màn hình chính}
    \end{subfigure}
    \hfill
    \begin{subfigure}[b]{0.23\textwidth}
        \centering
        \includegraphics[width=\textwidth]{dash2.jpg}
        \caption{Chi tiết chỉ số}
    \end{subfigure}
    \hfill
    \begin{subfigure}[b]{0.23\textwidth}
        \centering
        \includegraphics[width=\textwidth]{dash3.jpg}
        \caption{Biểu đồ}
    \end{subfigure}
    
    \vspace{1cm}
    
    \begin{subfigure}[b]{1.0\textwidth} 
        \centering
        \begin{subfigure}[b]{0.23\textwidth}
            \centering
            \includegraphics[width=\textwidth]{forgot_pass1.jpg}
            \caption{Đặt lại mật khẩu} 
        \end{subfigure}
        \hspace{0.5cm} 
        \begin{subfigure}[b]{0.23\textwidth}
            \centering
            \includegraphics[width=\textwidth]{forgot_pass2.jpg}
            \caption{Thông báo}   
        \end{subfigure}
        
    \end{subfigure}
    
    \caption{Màn hình Quên mật khẩu}
    \label{fig:dashboard_and_auth}
\end{figure}



\begin{figure}[H]
    \centering
    % --- Hình 1 ---
    \begin{subfigure}[b]{0.23\textwidth}
        \centering
        \includegraphics[width=\textwidth]{his.jpg}
        \caption{Lịch sử}
    \end{subfigure}
    \hfill
    % --- Hình 2 ---
    \begin{subfigure}[b]{0.23\textwidth}
        \centering
        \includegraphics[width=\textwidth]{hint.jpg}
        \caption{Trợ lý AI}
    \end{subfigure}
    \hfill
    % --- Hình 3 ---
    \begin{subfigure}[b]{0.23\textwidth}
        \centering
        \includegraphics[width=\textwidth]{data.jpg}
        \caption{Dữ liệu}
    \end{subfigure}
    \hfill
    % --- Hình 4 (Sẽ nhảy lên cùng hàng vì đã bỏ caption ở giữa) ---
    \begin{subfigure}[b]{0.23\textwidth}
        \centering
        % Lưu ý: kiểm tra xem có cần 'img/' không tùy cấu hình của bạn
        \includegraphics[width=\textwidth]{img/profile.jpg} 
        \caption{Hồ sơ}
    \end{subfigure}
    
    \caption{Các màn hình tiện ích chính}
    \label{fig:others}
\end{figure}

\begin{figure}[H]
    \centering
    
    % --- HÀNG 1: Chiều cao & Cân nặng ---
    
    % Cụm Chiều cao (Gồm 2 ảnh)
    \begin{subfigure}[b]{0.48\textwidth}
        \centering
        % Ảnh 1
        \includegraphics[width=0.48\linewidth]{data-height.jpg} 
        \hfill
        % Ảnh 2
        \includegraphics[width=0.48\linewidth]{data-height2.jpg}
        \caption{Nhập dữ liệu Chiều cao} % Caption chung cho 2 ảnh trên
    \end{subfigure}
    \hfill
    % Cụm Cân nặng (Gồm 2 ảnh)
    \begin{subfigure}[b]{0.48\textwidth}
        \centering
        \includegraphics[width=0.48\linewidth]{data-weight.jpg}
        \hfill
        \includegraphics[width=0.48\linewidth]{data-weight2.jpg}
        \caption{Nhập dữ liệu Cân nặng}
    \end{subfigure}

    \vspace{0.5cm} % Khoảng cách giữa hàng 1 và hàng 2

    % --- HÀNG 2: Nhịp tim & Giấc ngủ ---

    % Cụm Nhịp tim
    \begin{subfigure}[b]{0.48\textwidth}
        \centering
        \includegraphics[width=0.48\linewidth]{nt1.jpg}
        \hfill
        \includegraphics[width=0.48\linewidth]{nt2.jpg}
        \caption{Theo dõi Nhịp tim}
    \end{subfigure}
    \hfill
    % Cụm Giấc ngủ
    \begin{subfigure}[b]{0.48\textwidth}
        \centering
        \includegraphics[width=0.48\linewidth]{sl1.jpg}
        \hfill
        \includegraphics[width=0.48\linewidth]{sl2.jpg}
        \caption{Theo dõi Giấc ngủ}
    \end{subfigure}

    \vspace{0.5cm} % Khoảng cách giữa hàng 2 và hàng 3

    % --- HÀNG 3: Huyết áp & Mục tiêu ---

    % Cụm Huyết áp
    \begin{subfigure}[b]{0.48\textwidth}
        \centering
        \includegraphics[width=0.48\linewidth]{ha1.jpg}
        \hfill
        \includegraphics[width=0.48\linewidth]{ha2.jpg}
        \caption{Theo dõi Huyết áp}
    \end{subfigure}
    \hfill
    % Cụm Mục tiêu
    \begin{subfigure}[b]{0.48\textwidth}
        \centering
        \includegraphics[width=0.48\linewidth]{g1.jpg}
        \hfill
        \includegraphics[width=0.48\linewidth]{g2.jpg}
        \caption{Thiết lập Mục tiêu}
    \end{subfigure}

    % Caption tổng của cả Figure lớn
    \caption{Tổng hợp các màn hình nhập liệu và theo dõi sức khỏe}
    \label{fig:all_inputs_monitoring}
\end{figure}


%----------- CHƯƠNG 8: KHÓ KHĂN VÀ GIẢI PHÁP -----------
\chapter{KHÓ KHĂN VÀ GIẢI PHÁP}

\section{Khó khăn gặp phải}

\subsection{Học Jetpack Compose}
\textbf{Khó khăn:} Jetpack Compose là công nghệ mới, khác biệt hoàn toàn với XML-based UI truyền thống.

\textbf{Giải pháp:}
\begin{itemize}
    \item Học qua documentation chính thức của Google
    \item Xem các video tutorial và sample projects
    \item Thực hành với các composable nhỏ trước khi build full screen
\end{itemize}

\subsection{UI/UX Design}
\textbf{Khó khăn:} Thiết kế giao diện đẹp, trực quan và user-friendly.

\textbf{Giải pháp:}
\begin{itemize}
    \item Nghiên cứu Material Design 
    \item Tham khảo các ứng dụng tương tự
    \item Thiết kế mockup trên Figma trước
\end{itemize}

\subsection{Performance Optimization}
\textbf{Khó khăn:} App chậm khi load nhiều dữ liệu, xử lý đăng tốn nhiều thời gian.

\textbf{Giải pháp:}
\begin{itemize}
    \item Sử dụng LazyColumn thay vì Column cho danh sách dài
    
\end{itemize}

\section{Bài học kinh nghiệm}

\subsection{Quản lý dự án}
\begin{itemize}
    \item Lập kế hoạch chi tiết giúp tiết kiệm thời gian
    \item Sử dụng git để quản lý dự án
    \item Chia từng tính năng thành từng công việc nhỏ
\end{itemize}

\subsection{Làm việc nhóm}
\begin{itemize}
    \item Kỹ năng làm việc nhóm, phân chia công việc
    \item Viết tài liệu hướng dẫn dùng chung
    \item Khả năng giao tiếp và trao đổi, giải quyết vấn đề
\end{itemize}

%----------- CHƯƠNG 10: HƯỚNG PHÁT TRIỂN -----------
\chapter{HƯỚNG PHÁT TRIỂN}

\section{Tính năng mở rộng}

\subsection{Ngắn hạn (3-6 tháng)}
\begin{enumerate}
    \item \textbf{Hoàn thiện tính năng hiện tại}
    \begin{itemize}
        \item Hoàn thành các chức năng trong Menu tùy chọn ở các màn hình
        \item Hoàn thành chức năng Trợ lý AI
        \item Phát triển chức năng Tìm phòng tập gần đây
        \item Thêm tính năng thông báo (notification) và nhắc nhở (reminder)
    \end{itemize}
    
    \item \textbf{Tích hợp Smartwatch}
    \begin{itemize}
        \item Tích hợp Google Fit API
        \item Đồng bộ dữ liệu từ Wear OS
        \item Sync heart rate, steps từ smartwatch
    \end{itemize}
    
\end{enumerate}

\subsection{Dài hạn (1-2 năm)}
\begin{enumerate}
    \item \textbf{Multi-platform}
    \begin{itemize}
        \item Phát triển iOS app với SwiftUI hoặc Flutter
        \item Web application với React
        \item Desktop app (Windows, macOS)
    \end{itemize}
    
    \item \textbf{Healthcare Integration}
    \begin{itemize}
        \item Kết nối với bệnh viện/phòng khám
        \item Chia sẻ dữ liệu với bác sĩ
    \end{itemize}

\end{enumerate}



%----------- CHƯƠNG 11: KẾT LUẬN -----------
\chapter{KẾT LUẬN}

\section{Tổng kết}

Dự án \textbf{Salud - Ứng dụng Quản lý Sức khỏe} đã đạt được những kết quả:

\subsection{Về mặt kỹ thuật}
\begin{itemize}
    \item Xây dựng thành công ứng dụng Android với Kotlin và Jetpack Compose
    \item Áp dụng kiến trúc MVVM một cách hiệu quả
    \item Tích hợp Firebase Authentication và Firestore hoàn chỉnh
    \item Triển khai các tính năng core: Dashboard, Health Tracking, Nutrition Management
    \item Giao diện người dùng đẹp mắt, hiện đại và responsive
\end{itemize}

\subsection{Về mặt học tập}
Qua dự án này, nhóm đã:
\begin{itemize}
    \item Nắm vững Kotlin và Jetpack Compose
    \item Hiểu sâu về MVVM architecture pattern
    \item Có kinh nghiệm làm việc với Firebase platform
    \item Rèn luyện kỹ năng làm việc nhóm và giải quyết vấn đề
\end{itemize}

\subsection{Về mặt ứng dụng thực tế}
\begin{itemize}
    \item Ứng dụng giải quyết được nhu cầu thực tế của người dùng
    \item Có tiềm năng phát triển thành sản phẩm thương mại
    \item Giao diện thân thiện, dễ sử dụng cho mọi đối tượng
    \item Có thể mở rộng với nhiều tính năng nâng cao
\end{itemize}

\section{Đánh giá}

\subsection{Điểm mạnh}
\begin{itemize}
    \item Sử dụng công nghệ hiện đại (Kotlin, Compose, Firebase)
    \item Kiến trúc rõ ràng, dễ duy trì và mở rộng
    \item UI/UX đẹp mắt, trực quan, dễ sử dụng
    \item Tiện ích đa dạng, phù hợp với xu hướng hiện tại
    \item Hiệu năng tốt, ít lag
\end{itemize}

\subsection{Điểm cần cải thiện}
\begin{itemize}
    \item Một số tính năng chưa hoàn thiện (Menu tuỳ chọn, AI Chatbot)
    \item Chưa có tính năng offline hoàn chỉnh
    \item Chưa optimize cho tablet và foldable devices
\end{itemize}

\section{Lời cảm ơn}

Nhóm xin chân thành cảm ơn:
\begin{itemize}
    \item \textbf{Thầy Trương Quang Tuấn} - Giảng viên hướng dẫn, đã nhiệt tình hỗ trợ giảng dạy những kiến thức trọng tâm và thực tế trong môi trường làm việc.
    \item \textbf{Khoa Công nghệ Thông tin} - Đại học Giao thông Vận tải TP.HCM, đã tạo điều kiện về cơ sở vật chất và môi trường học tập.
\end{itemize}

\section{Kết luận cuối cùng}

Salud không chỉ là một đồ án môn học, mà còn là một sản phẩm có tiềm năng thực tế, giúp cải thiện chất lượng cuộc sống của người dùng. Nhóm cam kết sẽ tiếp tục phát triển và hoàn thiện ứng dụng trong tương lai, hướng tới mục tiêu đưa Salud trở thành một trong những ứng dụng quản lý sức khỏe hàng đầu tại Việt Nam.

Dự án đã giúp nhóm không chỉ nâng cao kiến thức chuyên môn mà còn rèn luyện được các kỹ năng mềm quan trọng như làm việc nhóm, quản lý thời gian, và tư duy giải quyết vấn đề - những kỹ năng cần thiết cho công việc thực tế sau này.

%----------- TÀI LIỆU THAM KHẢO -----------
\begin{thebibliography}{99}

\bibitem{android_docs}
Google Developers, 
\textit{Android Developer Documentation},
\url{https://developer.android.com/docs}, 
2024.

\bibitem{compose_docs}
Google Developers,
\textit{Jetpack Compose Documentation},
\url{https://developer.android.com/jetpack/compose},
2024.

\bibitem{kotlin_docs}
JetBrains,
\textit{Kotlin Programming Language},
\url{https://kotlinlang.org/docs/home.html},
2024.

\bibitem{firebase_docs}
Google Firebase,
\textit{Firebase Documentation},
\url{https://firebase.google.com/docs},
2024.

\bibitem{mvvm_pattern}
Microsoft,
\textit{Model-View-ViewModel Pattern},
\url{https://learn.microsoft.com/en-us/dotnet/architecture/maui/mvvm},
2023.

\bibitem{material_design}
Google,
\textit{Material Design Guidelines},
\url{https://m3.material.io/},
2024.

\bibitem{nutrition_science}
World Health Organization,
\textit{Nutrition and Food Safety},
\url{https://www.who.int/health-topics/nutrition},
2024.

\bibitem{bmi_calculation}
Centers for Disease Control and Prevention,
\textit{Body Mass Index (BMI)},
\url{https://www.cdc.gov/bmi/},
2024.

\bibitem{android_architecture}
Google Developers,
\textit{Guide to app architecture},
\url{https://developer.android.com/topic/architecture},
2024.

\bibitem{coroutines}
JetBrains,
\textit{Kotlin Coroutines Guide},
\url{https://kotlinlang.org/docs/coroutines-guide.html},
2024.

\bibitem{room_database}
Google Developers,
\textit{Room Persistence Library},
\url{https://developer.android.com/training/data-storage/room},
2024.

\bibitem{navigation_compose}
Google Developers,
\textit{Navigation with Compose},
\url{https://developer.android.com/jetpack/compose/navigation},
2024.

\bibitem{health_tracking}
Patel, M. S., Asch, D. A., \& Volpp, K. G.,
\textit{Wearable devices as facilitators, not drivers, of health behavior change},
JAMA, 2015.

\bibitem{mobile_health}
Steinhubl, S. R., Muse, E. D., \& Topol, E. J.,
\textit{The emerging field of mobile health},
Science Translational Medicine, 2015.

\bibitem{figma_design}
Figma Inc.,
\textit{Figma Design Tool},
\url{https://www.figma.com/},
2024.

\end{thebibliography}

%----------- PHỤ LỤC -----------
\appendix

\chapter{Mã nguồn quan trọng}

\section{HomeViewModel.kt}
\begin{lstlisting}[language=Java]
class HomeViewModel : ViewModel() {
    private val _uiState = MutableStateFlow(HomeUiState())
    val uiState: StateFlow<HomeUiState> = _uiState.asStateFlow()
    
    fun loadUserData() {
        viewModelScope.launch {
            try {
                val user = repository.getCurrentUser()
                _uiState.update { it.copy(user = user) }
            } catch (e: Exception) {
                _uiState.update { it.copy(error = e.message) }
            }
        }
    }
    
    fun calculateBMI(weight: Double, height: Double): Double {
        return weight / (height * height)
    }
}
\end{lstlisting}

\section{Navigation Setup}
\begin{lstlisting}[language=Java]
@Composable
fun MainNavigation() {
    val navController = rememberNavController()
    
    NavHost(navController, startDestination = "splash") {
        composable("splash") { SplashScreen(navController) }
        composable("login") { LoginScreen(navController) }
        composable("home") { HomeScreen(navController) }
        composable("data") { DataScreen(navController) }
        composable("diary") { DiaryScreen(navController) }
        composable("profile") { ProfileScreen(navController) }
    }
}
\end{lstlisting}


\chapter{Tài liệu hướng dẫn sử dụng}

\section{Cài đặt ứng dụng}
\begin{enumerate}
    \item Tải file APK từ \href{https://github.com/uthprj/Salud_App}{Github} hoặc tại \href{https://hiennam2803.github.io/Salud_download_web/}{Website}
    
    \begin{figure}[H] 
        \centering
        \includegraphics[width=0.3\textwidth]{img/salud_qr.png}
        \caption{Quét mã QR để tải nhanh ứng dụng}
        \label{fig:qr_code}
    \end{figure}

    \item Bật "Install from Unknown Sources" trên thiết bị Android
    \item Mở file APK và cài đặt
    \item Khởi động ứng dụng
\end{enumerate}

\section{Hướng dẫn sử dụng cơ bản}
\begin{enumerate}
    \item \textbf{Đăng ký tài khoản:} Nhập email và mật khẩu, nhấn Register
    \item \textbf{Đăng nhập:} Nhập thông tin đăng nhập
    \item \textbf{Cập nhật thông tin:} Vào Profile > Edit Profile
    \item \textbf{Nhập cân nặng:} Vào Data > Health > Weight > Add
    \item \textbf{Thêm bữa ăn:} Vào Data > Nutrition > Add Meal
    \item \textbf{Xem nhật ký:} Vào tab Diary
\end{enumerate}

\begin{center}
-- Hết --
\end{center}



\end{document}
